%%%%%%%%%%%%%%%%%%%%%%%%%%%%%%%%%%%%%%%%%%%%%%%%%%%%%%%%%%%%%%%%%%%%%%%%
%
% PS1QLS: Survey Paper
%
%%%%%%%%%%%%%%%%%%%%%%%%%%%%%%%%%%%%%%%%%%%%%%%%%%%%%%%%%%%%%%%%%%%%%%%%

\documentclass[useAMS,usenatbib]{mn2e}
%% letterpaper
%% a4paper

% \voffset=-0.8in

% Packages:
\input psfig.sty
\usepackage{xspace}
\usepackage{graphicx}

% Macros:
% JOURNALS
\newcommand {\apj} {ApJ}
\newcommand {\apjl} {ApJL}
\newcommand {\apjs} {ApJS}
\newcommand {\mnras} {MNRAS}
\newcommand {\apss} {Ap \& SS}
\newcommand {\aap} {A\&A}
\newcommand {\aj} {AJ}
\newcommand {\prd} {Phys. Rev. D}
\newcommand {\nat} {Nature}
\newcommand {\araa} {ARA\&A}
\newcommand {\jgr} {J. Geophys. Res.}
\newcommand {\pasp} {PASP}

% MISC
\newcommand {\etal} {et~al.~}
\def \spose#1{\hbox  to 0pt{#1\hss}}  
\newcommand {\lta} {\mathrel{\spose{\lower 3pt\hbox{$\sim$}}\raise  2.0pt\hbox{$<$}}}
\newcommand {\gta} {\mathrel{\spose{\lower  3pt\hbox{$\sim$}}\raise 2.0pt\hbox{$>$}}}
\def \ion#1#2{#1{\footnotesize{#2}}\relax}
\newcommand {\ha}  {\ifmmode H\alpha \else H$\alpha $ \fi} 
\newcommand {\hi} {\ion{H}{I} } 
% \newcommand {\Sref} {\S}
\def\Sref#1{Section~\ref{#1}\xspace}
\def\Fref#1{Figure~\ref{#1}\xspace}
\def\Tref#1{Table~\ref{#1}\xspace}
\def\Eref#1{Equation~\ref{#1}\xspace}
\def\Eqref#1{Eq.~(\ref{#1})\xspace}

% UNITS
\newcommand {\kms} {\ifmmode  \,\rm km\,s^{-1} \else $\,\rm km\,s^{-1}  $ \fi }
\newcommand {\kpc} {\ifmmode  {\rm kpc}  \else ${\rm  kpc}$ \fi  }  
\newcommand {\pc} {\ifmmode  {\rm pc}  \else ${\rm pc}$ \fi  }  
\newcommand {\Msun} {\ifmmode {\rm M_{\odot}} \else ${\rm M_{\odot}}$ \fi} 
\newcommand {\Zsun} {\ifmmode {\rm Z_{\odot}} \else ${\rm Z_{\odot}}$ \fi} 
\newcommand {\yr} {\ifmmode yr^{-1} \else $yr^{-1}$ \fi} 
\newcommand {\hMsun} {\ifmmode h^{-1}\,\rm M_{\odot} \else $h^{-1}\,\rm M_{\odot}$ \fi}

% COSMOLOGY
%\newcommand {\LCDM} {\ifmmode \Lambda{\rm CDM} \else $\Lambda{\rm CDM}$ \fi}

% LENSING
\def\zd{z_{\rm d}}
\def\zs{z_{\rm s}}
\def\zspdf{z_{\rm s,pdf}\;}
\def\Dd{D_{\rm d}}
\def\Ds{D_{\rm s}}
\def\Dds{D_{\rm ds}}
\def\Sigmacrit{\Sigma_{\rm crit}}
\def\REin{R_{\rm Ein}}
\def\MEin{M_{\rm Ein}}
\def\sigmasie{\sigma_{\mathrm{SIE}}}
\def\Mvir{M_{\rm vir}}
\def\Mhalo{M_{\rm h}}
\def\Vhalo{V_{\rm c,h}}
\def\rhalo{r_{\rm c,h}}
\def\qhalo{q_{3,\rm h}}
\def\vc{V_{\rm c}}
\def\rc{r_{\rm c}}
\def\q3{q_{3}}
\def\bsis{b_{\rm SIS}}

% SED/PHOTOMETRIC FITTING
\def\Mstar{M_{*}}
\def\logMstar{\log_{10}\left(\Mstar/\Msun\right)}
\def\Mstarb{M_{*,\rm b}}
\def\logMstarb{\log_{10}\left(\Mstarb/\Msun\right)}
\def\Mstard{M_{*,\rm d}}
\def\logMstard{\log_{10}\left(\Mstard/\Msun\right)}
% \def\Reff{R_{\rm eff}}
% \def\Reffb{R_{\rm eff,b}}
% \def\Reffd{R_{\rm eff,d}}
\def\Reff{R_{\rm 50}}
\def\Reffb{R_{\rm 50,b}}
\def\Reffd{R_{\rm 50,d}}
\def\sersic{S\'ersic}
\def\nb{n_{\rm b}}
\def\qb{q_{\rm b}}
\def\phib{phi_{\rm b}}
\def\nd{n_{\rm d}}
\def\qd{q_{\rm d}}
\def\phid{phi_{\rm d}}

% SOFTWARE/HARDWARE
\def\SExtractor{{\sc SExtractor}\xspace}
\def\hst{{\it HST}\xspace}
\def\acs{\hst/ACS\xspace}
\def\galfit{{\sc galfit}\xspace}
\def\idl{{\sc idl}\xspace}
\def\python{{\sc python}\xspace}
\def\SPASMOID{{\sc SPASMOID}\xspace}

% FILTERS
\def\Bfilter{F450W\xspace}
\def\Vfilter{F606W\xspace}
\def\Ifilter{F814W\xspace}
\def\Kfilter{K'\xspace}
\def\Bband{\Bfilter-band\xspace}
\def\Vband{\Vfilter-band\xspace}
\def\Iband{\Ifilter-band\xspace}
\def\Kband{\Kfilter-band\xspace}

% PROBABILITY THEORY
\def\pr{{\rm Pr}}
\def\data{{\mathbf{d}}}
\def\datap{{\mathbf{d}^{\rm p}}}
\def\datai{d_i}
\def\datapi{d^{\rm p}_i}
\def\masspars{\boldsymbol{\theta}_{\rm m}}
\def\srcpars{\boldsymbol{\theta}_{\rm s}}
\def\vrot{{\mathbf{v}}}
\def\vrotp{{\mathbf{v}^{\rm p}}}
\def\vrotmodel{\hat{\mathbf{v}}}
\def\vrotj{v_j}
\def\vrotpj{v^{\rm p}_j}

% SPECTROSCOPY
\def\Angstrom{A\xspace}
\def\CaII{Ca\,{\sc ii}\xspace}
\def\NaII{Na\,{\sc ii}\xspace}
\def\NaD{Na\,{\sc D}\xspace} % ??
\def\NII{N\,{\sc ii}\xspace}
\def\HeII{He\,{\sc ii}\xspace}
%\def\Mgb{Mg\,{\sc ii}} Mg II is something else, at 2800 A no?
\def\Mgb{Mg\,{\rm b}\xspace}
\def\Ha{H$\alpha$\xspace}
\def\Hb{H$\beta$\xspace}
\def\OII{[O\,{\sc ii}]\xspace}
\def\OIII{O\,{\sc iii}]\xspace}
\def\CIII{C\,{\sc iii}]\xspace}
\def\CIV{C\,{\sc iv}\xspace}
\def\FeII{[Fe\,{\sc ii}]\xspace}
\def\sigmap{$\sigma_{\rm ap}$}
\def\sigmasdss{$\sigma_{\rm SDSS}$}

% PAPER SERIES
\def\paperI{{Paper~I}\xspace}
\def\paperIfirsttime{{Treu \etal submitted, referred to hereafter as \paperI}\xspace}
\def\paperII{{Paper~II}\xspace}
\def\paperIIfirsttime{{Dutton \etal submitted, referred to hereafter as \paperII}\xspace}
\def\paperIII{{Paper~III}\xspace}
\def\paperIIIfirsttime{{Brewer \etal submitted, referred to hereafter as \paperIII}\xspace}

% SAMPLE PROPERTIES
\def\NSWELLS{27}
\def\NSWELLSNEW{19}
\def\NSWELLSHST{16}
\def\NSWELLSAO{3}
\def\NSWELLSA{8}
\def\NSWELLSB{1}
\def\NSWELLSC{6}
\def\NSWELLSX{4}
%10 spirals from slacs, including 8 ``edge-on''
\def\NSSLACS{10}
\def\NSESLACS{8}
\def\NTOTALA{16}

% COMMENTING
\usepackage[usenames]{color}
\newcommand{\aaron}[1]{\textcolor{Brown}{\bf #1}}
\newcommand{\phil}[1]{\textcolor{blue}{\bf #1}}
\newcommand{\brendon}[1]{\textcolor{Violet}{\bf #1}}
\newcommand{\tommaso}[1]{\textcolor{green}{\bf #1}}
\newcommand{\matt}[1]{\textcolor{orange}{\bf #1}}
\newcommand{\matteo}[1]{\textcolor{Magenta}{\bf #1}}
\newcommand{\flag}[1]{\textcolor{red}{\bf #1}}
\newcommand{\achtung}[2]{\textcolor{red}{\it\bf ATTENTION #1! #2}}

\def\oxford{Department of Physics, University of Oxford, Keble Road, Oxford, OX1 3RH, UK}
\def\mpia{Max Planck Institue for Astronomy, K\"onigstuhl 17, 69117, Heidelberg, Germany}
\def\nyu{Center for Cosmology and Particle Physics, New York University, NY, USA}

\def\pjmemail{\tt philip.marshall@astro.ox.ac.uk}
\def\epmemail{\tt morganson@mpia.de}

\newcommand\nodata{ ~$\cdots$~ }%


%%%%%%%%%%%%%%%%%%%%%%%%%%%%%%%%%%%%%%%%%%%%%%%%%%%%%%%%%%%%%%%%%%%%%%%%

\title[The PS1QLS Survey. I.]
{The PS1 Quasar Lens Survey. I. Candidate Selection}
    
\author[Some guys]{%
  Some guys, including
  Eric~P.~Morganson$^{1}$,
  Philip~J.~Marshall$^{2}$,
  David W. Hogg$^{3,1}$
  \newauthor{%
  and the PS1 Collaboration}
  \medskip\\
  $^1$\mpia\\
  $^2$\oxford\\
  $^3$\nyu
}

%%%%%%%%%%%%%%%%%%%%%%%%%%%%%%%%%%%%%%%%%%%%%%%%%%%%%%%%%%%%%%%%%%%%%%%%

\begin{document}
             
\date{to be submitted to MNRAS}
             
\pagerange{\pageref{firstpage}--\pageref{lastpage}}\pubyear{2010}

\maketitle           

\label{firstpage}

%%%%%%%%%%%%%%%%%%%%%%%%%%%%%%%%%%%%%%%%%%%%%%%%%%%%%%%%%%%%%%%%%%%%%%%%

\begin{abstract}

Gravitationally lensed quasars enable a wide range of astrophysical
investigations, most of which are currently limited by the available
sample size. The Pan-STARRS 1 quasar lens search (PS1QLS) aims to supply
this demand, using the 30,000 square degree, multi-filter, multi-epoch
``$3\pi$'' imaging dataset of the PS1 project.  In this paper we
describe: the preparation of the imaging and catalog data in the
PS1--SDSS DR7 overlap area, our target selection, using SDSS colors,
SDSS--PS1 variability and PS1 spatial extent, and our robotic approach
to the analysis of the targets. We show that our target selection
recovers N\% of the known lenses in the search area, and, using
realistic synthetic systems, should be X\% complete. 

\end{abstract}

% Full list of options at http://www.journals.uchicago.edu/ApJ/instruct.key.html

\begin{keywords}
  gravitational lensing
\end{keywords}

\setcounter{footnote}{1}

%%%%%%%%%%%%%%%%%%%%%%%%%%%%%%%%%%%%%%%%%%%%%%%%%%%%%%%%%%%%%%%%%%%%%%%%
%%  SECTION 1: INTRODUCTION
%%%%%%%%%%%%%%%%%%%%%%%%%%%%%%%%%%%%%%%%%%%%%%%%%%%%%%%%%%%%%%%%%%%%%%%%

\section{Introduction}
\label{sec:intro}


% Motivation.


% History. Future.


% Photometric quasar selection: Richards et al.


% Spatial extent: SQLS


% Follow-up: SQLS. PS1 as its own follow-up.


% Towards automated lens analysis: HAGGLes etc


% The PS1QLS series.

This paper is organized as follows... Throughout this paper, and the
rest of the PS1QLS series, magnitudes are given in the AB system
\citep{Oke74} and we adopt standard ``concordance'' cosmological
parameters, i.e. $h=0.7$, $\Omega_m=0.3$ and $\Omega_\Lambda=0.7$, where
the symbols indicate the Hubble Constant in units of 100 km s$^{-1}$
Mpc$^{-1}$ and the matter and dark energy density of the Universe in
units of the critical density \citep[e.g.\ ][]{Kom++09}.


%%%%%%%%%%%%%%%%%%%%%%%%%%%%%%%%%%%%%%%%%%%%%%%%%%%%%%%%%%%%%%%%%%%%%%%%
%%  SECTION X: ...
%%%%%%%%%%%%%%%%%%%%%%%%%%%%%%%%%%%%%%%%%%%%%%%%%%%%%%%%%%%%%%%%%%%%%%%%

\section{Section Heading}
\label{sec:xxx}


%%%%%%%%%%%%%%%%%%%%%%%%%%%%%%%%%%%%%%%%%%%%%%%%%%%%%%%%%%%%%%%%%%%%%%%%
%%  SECTION X: CONCLUSIONS
%%%%%%%%%%%%%%%%%%%%%%%%%%%%%%%%%%%%%%%%%%%%%%%%%%%%%%%%%%%%%%%%%%%%%%%%


\section{Conclusions}
\label{sec:conclusions}

In summary, ...

\begin{itemize}

\item We ...

\item The ...

\item The ...

\end{itemize}



%%%%%%%%%%%%%%%%%%%%%%%%%%%%%%%%%%%%%%%%%%%%%%%%%%%%%%%%%%%%%%%%%%%%%%%%
%%  ACKNOWLEDGMENTS
%%%%%%%%%%%%%%%%%%%%%%%%%%%%%%%%%%%%%%%%%%%%%%%%%%%%%%%%%%%%%%%%%%%%%%%%

\section*{Acknowledgments}
 
We thank XXX for useful discussions and suggestions.

EPM acknowledges ....
%
%
PJM was given support by the Royal 
Society in the form of a research fellowship.
%
DWH acknowledges support from ...
% 
PS1 acknowledgment.
%
Funding for the SDSS and SDSS-II was provided by the Alfred P. Sloan
Foundation, the Participating Institutions, the National Science
Foundation, the U.S. Department of Energy, the National Aeronautics
and Space Administration, the Japanese Monbukagakusho, the Max Planck
Society, and the Higher Education Funding Council for England. The
SDSS was managed by the Astrophysical Research Consortium for the
Participating Institutions. The SDSS Web Site is http://www.sdss.org/.



%%%%%%%%%%%%%%%%%%%%%%%%%%%%%%%%%%%%%%%%%%%%%%%%%%%%%%%%%%%%%%%%%%%%%%%%
%%  APPENDICES
% %%%%%%%%%%%%%%%%%%%%%%%%%%%%%%%%%%%%%%%%%%%%%%%%%%%%%%%%%%%%%%%%%%%%%%
% 
% \appendix
% 
% \section{}
% \label{sec:appendix}
% 
% 
%%%%%%%%%%%%%%%%%%%%%%%%%%%%%%%%%%%%%%%%%%%%%%%%%%%%%%%%%%%%%%%%%%%%%%%%
%%  REFERENCES
%%%%%%%%%%%%%%%%%%%%%%%%%%%%%%%%%%%%%%%%%%%%%%%%%%%%%%%%%%%%%%%%%%%%%%%%

% MNRAS does not use bibtex, input .bbl file instead. 
% Generate this in the makefile using bubble script in scriptutils:

%   bubble -f PS1QLS-survey.tex references.bib 

\bibliographystyle{apj}
\bibliography{references}
%\input{PS1QLS-survey.bbl}

%%%%%%%%%%%%%%%%%%%%%%%%%%%%%%%%%%%%%%%%%%%%%%%%%%%%%%%%%%%%%%%%%%%%%%%%

\label{lastpage}
\bsp

\end{document}

%%%%%%%%%%%%%%%%%%%%%%%%%%%%%%%%%%%%%%%%%%%%%%%%%%%%%%%%%%%%%%%%%%%%%%%%
