%%%%%%%%%%%%%%%%%%%%%%%%%%%%%%%%%%%%%%%%%%%%%%%%%%%%%%%%%%%%%%%%%%%%%%%%
%
% PS1QLS 1: UVX Search, Candidate Selection
%
%%%%%%%%%%%%%%%%%%%%%%%%%%%%%%%%%%%%%%%%%%%%%%%%%%%%%%%%%%%%%%%%%%%%%%%%

\documentclass[useAMS,usenatbib]{mn2e}
%% letterpaper
%% a4paper

% \voffset=-0.8in

% Packages:
\input psfig.sty
\usepackage{xspace}
\usepackage{graphicx}

% Macros:
\input{macros.tex}
\def\oxford{Department of Physics, University of Oxford, Keble Road, Oxford, OX1 3RH, UK}
\def\mpia{Max Planck Institue for Astronomy, K\"onigstuhl 17, 69117, Heidelberg, Germany}
\def\nyu{Center for Cosmology and Particle Physics, New York University, NY, USA}

\def\pjmemail{\tt philip.marshall@astro.ox.ac.uk}
\def\epmemail{\tt morganson@mpia.de}

\newcommand\nodata{ ~$\cdots$~ }%


%%%%%%%%%%%%%%%%%%%%%%%%%%%%%%%%%%%%%%%%%%%%%%%%%%%%%%%%%%%%%%%%%%%%%%%%

\title[The PS1QLS Survey. I.]
{The PS1 Quasar Lens Survey. I. Target Selection in the SDSS DR8 Area}
    
\author[Some guys]{%
  Some guys, including
  Eric~P.~Morganson$^{1}$,
  Philip~J.~Marshall$^{2}$,
  David W. Hogg$^{3,1}$
  \newauthor{%
  and the PS1 Collaboration}
  \medskip\\
  $^1$\mpia\\
  $^2$\oxford\\
  $^3$\nyu
}

%%%%%%%%%%%%%%%%%%%%%%%%%%%%%%%%%%%%%%%%%%%%%%%%%%%%%%%%%%%%%%%%%%%%%%%%

\begin{document}
             
\date{to be submitted to MNRAS}
             
\pagerange{\pageref{firstpage}--\pageref{lastpage}}\pubyear{2010}

\maketitle           

\label{firstpage}

%%%%%%%%%%%%%%%%%%%%%%%%%%%%%%%%%%%%%%%%%%%%%%%%%%%%%%%%%%%%%%%%%%%%%%%%

\begin{abstract}

Gravitationally lensed quasars enable a wide range of astrophysical
investigations, most of which are currently limited by the available sample
size. The Pan-STARRS 1 quasar lens search (PS1QLS) aims to supply this demand,
using the 30,000 square degree, multi-filter, multi-epoch ``$3\pi$'' imaging
dataset of the PS1 project.  In this paper we describe the preparation of the
imaging and catalog data in the PS1--SDSS DR7 overlap area, and our target
selection, using SDSS colors, SDSS--PS1 variability and PS1 spatial extent.
These targets will  be analyzed robotically in a companion paper.  We show
that our target selection recovers N\% of the known lenses in the search area,
and, using realistic synthetic systems, should be X\% complete. 

\end{abstract}

% Full list of options at http://www.journals.uchicago.edu/ApJ/instruct.key.html

\begin{keywords}
  gravitational lensing
\end{keywords}

\setcounter{footnote}{1}

%%%%%%%%%%%%%%%%%%%%%%%%%%%%%%%%%%%%%%%%%%%%%%%%%%%%%%%%%%%%%%%%%%%%%%%%
%%  SECTION 1: INTRODUCTION
%%%%%%%%%%%%%%%%%%%%%%%%%%%%%%%%%%%%%%%%%%%%%%%%%%%%%%%%%%%%%%%%%%%%%%%%

\section{Introduction}
\label{sec:intro}


% Motivation.


% History. Future.


% Photometric quasar selection: Richards et al. 2009 and others

% RICHARDS ET AL 2009

% 10 LINES OF ERIC'S NOTES:










% 10 LINES OF PHIL'S NOTES:










% Spatial extent: SQLS - Oguri, Inada papers. Close pair searches.

% OGURI ET AL 2006

% 10 LINES OF ERIC'S NOTES:










% 10 LINES OF PHIL'S NOTES:











% Follow-up: SQLS. PS1 as its own follow-up.


% Towards automated lens analysis: HAGGLes etc


% The PS1QLS series.

This paper is organized as follows... Throughout this paper, and the
rest of the PS1QLS series, magnitudes are given in the AB system
\citep{Oke74} and we adopt standard ``concordance'' cosmological
parameters, i.e. $h=0.7$, $\Omega_m=0.3$ and $\Omega_\Lambda=0.7$, where
the symbols indicate the Hubble Constant in units of 100 km s$^{-1}$
Mpc$^{-1}$ and the matter and dark energy density of the Universe in
units of the critical density \citep[e.g.\ ][]{Kom++09}.


%%%%%%%%%%%%%%%%%%%%%%%%%%%%%%%%%%%%%%%%%%%%%%%%%%%%%%%%%%%%%%%%%%%%%%%%
%%  SECTION 2: DATABASE
%%%%%%%%%%%%%%%%%%%%%%%%%%%%%%%%%%%%%%%%%%%%%%%%%%%%%%%%%%%%%%%%%%%%%%%%

\section{The PS1$+$SDSS DR8 Dataset}
\label{sec:data}

% ERIC: Introduce PS1 survey. Introduce SDSS survey. 
% Catalog: Uber cal. 
% Relevant quantities made by others. New quantites added by Eric. 
% Imaging: calibration, properties: depth and variance, IQ and variance.


%%%%%%%%%%%%%%%%%%%%%%%%%%%%%%%%%%%%%%%%%%%%%%%%%%%%%%%%%%%%%%%%%%%%%%%%
%%  SECTION 3: SIMULATED LENS SAMPLE
%%%%%%%%%%%%%%%%%%%%%%%%%%%%%%%%%%%%%%%%%%%%%%%%%%%%%%%%%%%%%%%%%%%%%%%%

\section{Reference Quasar Lenses}
\label{sec:ref}

\subsection{Known Quasar Lenses in the DR8 Area}
\label{sec:ref:known}

% PHIL: SQLS sample, review. 

\subsection{Simulated Quasar Lenses}
\label{sec:ref:mock}

% PHIL: Introduce OM10 catalog, assumptions made. Coloring. Adding
% variability. Mock SDSS catalog by analytic means. Assumption of giant photo 
% apertures justified by low IQ.

% ERIC: Adding to imaging data. IPP extraction and photometry.


%%%%%%%%%%%%%%%%%%%%%%%%%%%%%%%%%%%%%%%%%%%%%%%%%%%%%%%%%%%%%%%%%%%%%%%%
%%  SECTION 3: UVX OBJECTS
%%%%%%%%%%%%%%%%%%%%%%%%%%%%%%%%%%%%%%%%%%%%%%%%%%%%%%%%%%%%%%%%%%%%%%%%

\section{Photometric Quasar-like Object Selection}
\label{sec:uvx}

% Motivate UVX selection. Recall non-point source. 
% Describe box, including flux limit. Results, cf ref.
% Optimisation using ref lenses.
% Compare numbers to Richards et al. Completeness, rejection rate.


%%%%%%%%%%%%%%%%%%%%%%%%%%%%%%%%%%%%%%%%%%%%%%%%%%%%%%%%%%%%%%%%%%%%%%%%
%%  SECTION 4: EXTENT & VARIABILITY
%%%%%%%%%%%%%%%%%%%%%%%%%%%%%%%%%%%%%%%%%%%%%%%%%%%%%%%%%%%%%%%%%%%%%%%%

\section{Lens Candidate Selection by Spatial Extent and Variability}
\label{sec:uvx}

% Motivate joint selection. Kochanek, Pindor work. 
% Description of available quantities: extent, nsigma, astrometric var.
% Stacked catalog? Or avergae?
% Optimisation using ref lenses.
% Results on database, properties of sample as function of magnitude. 
% Completeness, rejection rate.

% Checking spatial extent. 


%%%%%%%%%%%%%%%%%%%%%%%%%%%%%%%%%%%%%%%%%%%%%%%%%%%%%%%%%%%%%%%%%%%%%%%%
%%  SECTION 5: DISCUSSION
%%%%%%%%%%%%%%%%%%%%%%%%%%%%%%%%%%%%%%%%%%%%%%%%%%%%%%%%%%%%%%%%%%%%%%%%

\section{Discussion}
\label{sec:discuss}

% Feasibility of 

%%%%%%%%%%%%%%%%%%%%%%%%%%%%%%%%%%%%%%%%%%%%%%%%%%%%%%%%%%%%%%%%%%%%%%%%
%%  SECTION 6: CONCLUSIONS
%%%%%%%%%%%%%%%%%%%%%%%%%%%%%%%%%%%%%%%%%%%%%%%%%%%%%%%%%%%%%%%%%%%%%%%%

\section{Conclusions}
\label{sec:conclusions}

In summary, ...

\begin{itemize}

\item We ...

\item The ...

\item The ...

\end{itemize}



%%%%%%%%%%%%%%%%%%%%%%%%%%%%%%%%%%%%%%%%%%%%%%%%%%%%%%%%%%%%%%%%%%%%%%%%
%%  ACKNOWLEDGMENTS
%%%%%%%%%%%%%%%%%%%%%%%%%%%%%%%%%%%%%%%%%%%%%%%%%%%%%%%%%%%%%%%%%%%%%%%%

\section*{Acknowledgments}
 
We thank XXX for useful discussions and suggestions.

EPM acknowledges ....
%
%
PJM was given support by the Royal 
Society in the form of a research fellowship.
%
DWH acknowledges support from ...
% 
PS1 acknowledgment.
%
Funding for the SDSS and SDSS-II was provided by the Alfred P. Sloan
Foundation, the Participating Institutions, the National Science
Foundation, the U.S. Department of Energy, the National Aeronautics
and Space Administration, the Japanese Monbukagakusho, the Max Planck
Society, and the Higher Education Funding Council for England. The
SDSS was managed by the Astrophysical Research Consortium for the
Participating Institutions. The SDSS Web Site is http://www.sdss.org/.



%%%%%%%%%%%%%%%%%%%%%%%%%%%%%%%%%%%%%%%%%%%%%%%%%%%%%%%%%%%%%%%%%%%%%%%%
%%  APPENDICES
% %%%%%%%%%%%%%%%%%%%%%%%%%%%%%%%%%%%%%%%%%%%%%%%%%%%%%%%%%%%%%%%%%%%%%%
% 
% \appendix
% 
% \section{}
% \label{sec:appendix}
% 
% 
%%%%%%%%%%%%%%%%%%%%%%%%%%%%%%%%%%%%%%%%%%%%%%%%%%%%%%%%%%%%%%%%%%%%%%%%
%%  REFERENCES
%%%%%%%%%%%%%%%%%%%%%%%%%%%%%%%%%%%%%%%%%%%%%%%%%%%%%%%%%%%%%%%%%%%%%%%%

% MNRAS does not use bibtex, input .bbl file instead. 
% Generate this in the makefile using bubble script in scriptutils:

%   bubble -f PS1QLS-survey.tex references.bib 

\bibliographystyle{apj}
\bibliography{references}
%\input{PS1QLS-survey.bbl}

%%%%%%%%%%%%%%%%%%%%%%%%%%%%%%%%%%%%%%%%%%%%%%%%%%%%%%%%%%%%%%%%%%%%%%%%

\label{lastpage}
\bsp

\end{document}

%%%%%%%%%%%%%%%%%%%%%%%%%%%%%%%%%%%%%%%%%%%%%%%%%%%%%%%%%%%%%%%%%%%%%%%%
